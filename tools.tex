%===================================== CHAPTER 7 Tools =================================

\chapter{Tools}

This section briefly describes all the tools used for this project, which includes development tools, communication tools and any additional tools.

\section{Development tools}

\subsection{Front end}
\begin{itemize}
	\item Ionic \cite{es1} - Ionic is a front end UI framework designed to assist the development of hybrid mobile applications. By using this framework it became easy for the team to speed up the design of the interface and test the application both on computers and devices. More details about Ionic can be found in \textbf{Section \ref{subsec:ionic}}
	\item PhoneGap (Apache Cordova) \cite{RA2} - PhoneGap is a framework that enables software developers to automatically wrap HTML5, CSS and JavaScript code into platform-specific code that can run on devices such as iOS and Android phones and tablets. The Ionic framework is based on using PhoneGap for compiling its code. More details about PhoneGap can be found in \textbf{Section \ref{subsec:phonegap}}
	\item Android Studio \cite{es22} - Android Studio is the official IDE for Android application development, it was necessary to have this installed in order to develop our application on android devices. It also provided a way to install various plugins that were useful or needed for the project.
	\item Node.js \cite{es23} - Node.js is a platform built on Chrome's JavaScript runtime for easily building fast, scalable network applications. This was another tool that was necessary to install, because PhoneGap is built on it. First released in May of 2009, Node.js has been gaining much popularity as a server-side platform. 
	\item Gulp.js \cite{es24} - Gulp is a build tool that is used as a part of PhoneGap in order to automate many common tasks, such as build processes and plugin handling.
	\item Sass \cite{es25} - Sass is an extension to CSS which adds functionality such as being able to use variables, nested rules and inline imports. Sass helps keep stylesheets organized, and is fully compatible with regular CSS syntax. Sass is the preferred tool for handling stylesheets in Ionic.
	\item Proto.io \cite{protoIO} - This a prototyping framework aimed for mobile apps, and it allowed the team to make very quick and functional prototypes that were used for user testing, both by the team and by the customer. When discussing design solutions, it was much faster and simpler to make revisions to the prototype than it would be to redesign the application itself. 
	\item Balsamiq \cite{es3} - Wireframing and mock-up tool which was used to create early mock-ups of the different user interface views. This was used because it allowed the team to quickly make wireframes for the interfaces, and it gives a more professional look than by  just making paper prototypes.
	\item Icomoon \cite{es4} - This is an application with the purpose of generating an icon font from svg files that you upload to it. You can also resize, adjust positions, and set default pixel sizes for the icons. The application turns all the icons into crisp-looking and easy scalable icons. It also automatically generates the HTML and CSS code which you can use to integrate the code into your own application. In this project, Icomoon was used to make the category icons that show the various categories on each story
	\item FontForge \cite{es5} - After creating an icon font with Icomoon, FontForge was used to make manual adjustments to the icons themselves. FontForge is an editor with many advanced options for giving icons a smoother look and symmetry.
	\item Karma \cite{KF3} - Karma is a test runner for AngularJS. With Karma it is possible to write JavaScript tests and run it in different browsers on both desktops and mobile phones. It is easy to run a test for every integration you make. More details about how Karma was used in this project can be found in \textbf{Section \ref{sec:unit_testing}}	
\end{itemize}

\subsection{Back end}
\begin{itemize}
	\item Docker \cite{EHW2} - Docker is an open platform that provides the possibility for system developers to build their application and deploy it to other computers or servers, which can then run the same application, unchanged. The motivation for using Docker was mainly so the team could run the developed application on SINTEF's own server. More details about Docker is described in \textbf{Section \ref{subsec:docker}}
	\item MySQL \cite{es8} - MySQL is one of the most widely used open source databases. It has many advantages when it comes to scalability and flexibility, and it is well suited for many types of application development. More details about this project's database and its design can be found in \textbf{Section \ref{sec:database_design}}
	\item Digitalt musem API \cite{digitaltMuseum} - The source of the stories in the application was Digitalt Fortalt, and in order to access these it was necessary to use Digitalt museum's API. The API is documented on its website and was incorporated into the project's database. Details about the API and the use of it can be found in \textbf{Section \ref{sec:harvesting}}
	\item PHPUnit \cite{KF2} - PHPUnit is a framework for writing and performing unit tests on PHP code. More details about how PHPUnit was used in this project can be found in \textbf{Section \ref{sec:unit_testing}}
	\item JUnit \cite{jUnit} - JUnit is a framework for doing unit testing in Java. 
	\item DbUnit \cite{dbUnit} - DbUnit is an extension of JUnit that can be used to test methods accessing a database. Among its advantages is the ability to put the database in a known state before each test, which gives the tester control over what to expect as results of tests that change the contents of the database.
	\item Mahout \cite{as9} \todo{Mahout krever at man har Apache Commons Math, Guava-libraries og slf4j. Skal disse nevnes?} - Mahout is a project that provides scalable machine learning algorithms. Mahout is primarily focused on algorithms for collaborative filtering, clustering and classification. How Mahout has been used in this project is described in \textbf{Section \ref{sec:personalization_how}}.
\end{itemize}

\section{Communication tools}
\label{sec:communication tools}

\begin{itemize}
	\item Google Drive \cite{es10} - Used for creating and sharing documents with the whole group, as well as editing shared documents in real-time. it also made it possible to share all files, whether it was images/diagrams/spreadsheets, or anything else. That was why the team decided to use this as a good solution for sharing all documentation.
	\item Dropbox \cite{es11} - Used to share documents between the group and the customer. This tool was used upon request by the customer.
	\item Facebook \cite{es13} - Used for discussions and notifications of various things such as meeting times. This was used because Facebook is something everyone was already familiar with, and something that most of the team members use on a regular basis, so messages would be quickly noticed.
	\item Trello \cite{es15} - Task management application, this was a handy way to quickly see what needs to be done and what has already been completed, similar to a scrum task board . Because of this, the team found it to be a good tool to use, as it speeds up the process of managing work tasks and gave a better overview of the progress.
\end{itemize}

\section{Additional tools}
\label{sec:additional_tools}

\begin{itemize}
	\item GitHub \cite{es12} - Used for making a code repository to be shared by the group while developing the system. The customer requested the use of GitHub, and it was also used because it seemed like the easiest way to share and implement code, as well as sharing the code with the customer.
	\item Draw.io \cite{es14} - This tool was the primary way of making diagrams and models, such as use-cases and WBS chart. This tool was chosen for this because it provides a lot of templates for different types of graphs, and as everyone uses the same tool for all diagrams the report achieves a consistent style for every diagram.
	\item Ganttify \cite{RG1} - Converting Trello boards into Gantt Charts, makes the process of creating a gantt chart and milstone plan easier and faster.
\end{itemize}

\cleardoublepage