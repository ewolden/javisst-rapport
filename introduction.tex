%===================================== CHAPTER 1 Introduction =================================

\chapter{Intro}

This chapter introduces the customer, the team, and the project’s definition and purpose.

\section{Stakeholders}

\subsection{Customer}

The employer for this research project was SINTEF, in this case represented by Jacqueline Floch and Shanshan Jiang. SINTEF is an independent multidisciplinary research organization within technology, science, social science and medicine. The organization has also provided assignments for the course IT2901 in the past.

\subsection{Team}

Table 1.xx lists the persons in charge of developing the project in this report, and some of their background competencies.

TODO sett inn Table 1.xx team description\newline

Within the team there was a good amount of general experience with web design and computer application design. Some members had experience with making databases, resulting in not having to dedicate extra time for researching this topic. There was a mix of valuable experience between front end and back end development, as well as knowledge about project management and how to relate to various actors such as users and other stakeholders.

\section{Project description}

In the course IT2901 [es20], Informatics Project II, at Norwegian University of Science and Technology (NTNU), the main assignment was to develop a software project for a customer, and was completed during the spring semester of 2015. The goal of the course was to gain practical experience with the development of a software process for a customer, covering the whole life-cycle of the software project.

The project in this report is named Personalized storytelling. The purpose of this project was to create a cross-platform application (iOS and Android) which would allow users to discover personalized cultural and historical stories based on context-sensitive information and personal interests. The application is a part of the TagCloud [es21] project, which is a project whose aim is enriching cultural and historical experiences through innovative mobile applications.
There have been similar projects worked on in the past, one example being stedr, which is an application based on discovering and sharing stories about  locations around you (see chapter 3.3.1 for details about stedr). This project was different in the way that it was directly connected to the personalization of stories, and as such, multiple ways of personalization had to be integrated to find a good recommendation for the user.

\section{Problem description}

Even though there is a rich cultural heritage in Norway, many people do not participate in cultural activities. Museums and other cultural institutions have tried to increase interest with innovative exhibitions and tools, but with little luck. The motivation behind this project was to find out how effective personalization is in engaging more people in their cultural heritage. This was done by creating an application that picks out personalized stories based on the user's interests and context, and will thus encourage exploration and find relevant and interesting stories to the user. 

The target audience for the application included both those who have an interest in cultural activities, and also those who do not have any interest or experience about this, so as to encourage more of the general population to discover an interest in the subject. A specific target audience of 16-19 year old teenagers were promoted as a possible focus, because of the possibility of encouraging a young audience to develop an interest in cultural heritage.

To summarize, this report details the entire development process of a mobile application, developed for Android and iOS. The application will provide its users with cultural stories in a personalized manner, with the goal of generating more interest in cultural heritage. In the next chapter, a thorough pre-study of all the relevant technologies and methodologies that were considered to accomplish this task will be presented.

\cleardoublepage