%===================================== CHAPTER 4 Risk management =================================

\chapter{Risk management}

This chapter details the risk management of the project, which includes planning and handling all the various potential risks to the project.\newline

The risk analysis below contains a list of possible occurrences that could be harmful to the project. Provided for each risk is a short description, an estimated likelihood that the risk will happen, an estimated impact to the project if it happens, the importance of the risk, a preventive action to try to avoid the problem and a remedial action if the problem were to occur. Likelihood and impact estimates were rated on a scale from 1 to 9, with 9 being the highest, and the importance was calculated by multiplying likelihood with impact. The risk list was updated regularly and sorted by the importance value, thus the risk to be most aware of at each stage of the development process was at the top of the list.

\begin{table}[!h]
	\small
	\centering
		\begin{tabular}{ | p{3.5cm} | p{2cm} | p{1.5cm} | p{2cm} | p{3.5cm} | p{3.5cm} |}
			\hline
			\textbf{Description} & \textbf{Likelihood (1-9)} & \textbf{Impact (1-9)} & \textbf{Importance (Likelihood * Impact)} & \textbf{Preventive action} & \textbf{Remedial action} \\ \hline
			
			Underestimate the time planned to use for assignments & 8 & 8 & 64 & Estimate a little higher. continuous meetings. Continuous status update on tasks & Extra work hours and help each other. Have a clear prioritization of tasks so that some less important tasks can be delayed if needed. \\ \hline
			
			The group does not deliver updated information on the report and cannot maximize the quality of the feedback obtained from the supervisor & 6 & 7 & 42 & Always make sure the report meet the demands upon delivery & Ask concrete questions to the supervisor or other competent acquaintances of the group members \\ \hline
			
			An issue in the code that is not understood or can not be fixed. & 5 & 8 & 40 & Comment on the code, and talk with each other about the work done on the code & Get help from the supervisor or other people involved. \\ \hline
			
			The group does not receive quality feedback from supervisor & 6 & 6 & 36 & Be prepared for supervisor meetings. Prepare concrete questions and discuss issues with supervisor & Ask qualified acquaintances to read and give feedback on the report.  \\ \hline
			
			Project complexity / Project too difficult & 5 & 5 & 25 & Do not plan many complicated tasks & Downgrade demands \\ \hline
			
			Poor communication with the customer leading to misunderstandings and doubts about the progress of the project & 4 & 6 & 24 & Be well prepared for meetings by establishing an agenda for the meeting and sharing it with the customer beforehand. Establish a good customer relationship. Send email to the customer for clarification & Send email to the customer for clarification. Cooperate with customer to reprioritize tasks \\ \hline
			
			Poor communication within the group leading to misunderstandings and doubts about the progress of the project & 4 & 6 & 24 & Write meeting minutes to document decisions. Have frequent meetings where every team member explains what they have done and what they are planning to do. & Make a group decision to solve the misunderstanding \\ \hline
			
			Wrong choice of development tools & 4 & 6 & 24 & Do thorough research before deciding which tools to work with & If early in project, consider changing tools \\ \hline
			
			Absence of group member(s) over a period of time & 6 & 4 & 24 & Every member of the group should be aware of what all members are working on, so that they can step in and take over the absent person's tasks & When the progress is halted by a person's absence, other group members should take over the tasks needed for further progress \\ \hline
			
			The workload is distorted. Some members of the group work too much, while others work too little. & 5 & 4 & 20 & Continuous meetings, after every meeting the team members delegate assignments. The leader of the meeting also have to make sure that everyone gets approximately the same amount of work. & Redelegate work tasks within the group. \\ \hline
			
			Poor choice of programming language. It becomes difficult to produce the product before deadline & 2 & 9 & 18 & Good research and discussion in the group. Do not choose a language that some people in the group do not have experience with & Reevaluate non-functional requirements with the customer \\ \hline
			
			Customer changes requirements & 6 & 3 & 18 & Constant communication with customer & Use an agile development process to better adapt to changes \\ \hline
			
			Data loss & 2 & 8 & 16 & Local copy and regular backups & Restore latest available backup \\ \hline
			
			Disagreement within the group on key issues in the project & 3 & 5 & 15 & Establish ground rules regarding how to discuss key issues and how to make decisions final. Make democratic decisions & If the disagreement cannot be solved, one may involve the supervisor \\ \hline
			
			Personal conflicts between group members & 2 & 7 & 14 & Establish ground rules for the team so that emerging conflicts are solved as early as possible & First try to solve the conflict between the group members in question. If this doesn't work, involve the whole group. A last resort would be to involve the supervisor \\ \hline
			
			Product doesn't meet requirements & 2 & 7 & 14 & Request product feedback from customer & Assess which changes must be made and prioritize the most important parts to change \\ \hline
			
			Overestimate the time planned to use for assignments & 3 & 4 & 12 & Investigate the assignments so it is clear what they include, and how much effort it would take to perform them. & When the assignment is done - long before the time, use the extra time on the more time consuming assignments. \\ \hline
			
			Missing deadlines & 1 & 8 & 8 & Have frequent meetings and plan well & Do extra work hours to finish the work as quickly as possible \\ \hline
			
			Product is not user-friendly & 3 & 2 & 6 & Perform usability tests & Assess which changes must be made and prioritize the most important parts to change \\ \hline
			
			Technical issues(server down) & 2 & 3 & 6 & & Use the backup \\ \hline
		\end{tabular}
	\caption{Risk list}
	\label{Tab_risks}
\end{table}

\cleardoublepage