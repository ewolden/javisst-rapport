%===================================== CHAPTER 4 Risk management =================================

\chapter{Risk management}
\label{chap:risk_management}

This chapter details the risk management of the project, which includes planning and handling all the various potential risks to the project.\newline

The risk analysis below contains a list of possible occurrences that could be harmful to the project. Provided for each risk is a short description, an estimated likelihood that the risk will happen, an estimated impact to the project if it happens, the importance of the risk, a preventive action to try to avoid the problem and a remedial action if the problem were to occur. Likelihood and impact estimates were rated on a scale from 1 to 9, with 9 being the highest, and the importance was calculated by multiplying likelihood with impact. The risk list was updated regularly and sorted by the importance value, thus the risk to be most aware of at each stage of the development process was at the top of the list.

\begin{center}
	\begin{longtable}{ | p{3.5cm} | p{2cm} | p{1.5cm} | p{2cm} | p{3.5cm} | p{3.5cm}|}
		
		\caption[Risk list example]{Risk list example } \label{Tab:riskexample}\\
		\hline
		\textbf{Description} & \textbf{Likelihood(1-9)} & \textbf{Impact(1-9)} & \textbf{Importance (Likelihood * Impact)} & \textbf{Preventive action} & \textbf{Remedial action}\\ \hline
		
		Customer changes requirements & 6 & 3 & 18 & Constant comunication with customer & Use an agile development process to better adapt to changes \\\hline 
		
		Poor communication within the group leading to misunderstandings and doubts about the progress of the project & 4 & 6 & 24 & Write meeting minutes to document decisions. Have frequent meetings where every team member explains what they have done and what they are planning to do & Make a group decision to solve the misunderstanding \\ \hline
	\end{longtable}
\end{center}
		

\cleardoublepage