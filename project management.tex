%===================================== CHAPTER 3 Project management =================================

\chapter{Project management}

This chapter describes how the team delegated work tasks and responsibilities, and also how the work was broken down into manageable tasks. The chapter also describes time management and quality assurance for the project.

\section{Scrum team and roles}

The role delegation in the team is detailed in table 3.xx below. The delegation of roles was primarily based on personal interest and motivation. The tasks were divided up into main responsibility areas for back end and front end before assigning people to each one. However, this was only  a guideline for main responsibilities, and the group members had to be flexible and work on some tasks outside of their main areas.

TODO sett inn Table 3.xx role delegation\newline

\section{Work breakdown structure}

A work breakdown structure is a decomposition of the project and its goal is to break down each part of the development process into manageable parts to ease the planning and execution of the development. Each element in the diagram can be a product, data, service or a combination of the three. One of the benefits of detailing a project this way appears when doing cost estimation and scheduling the team around the project (i.e. should ease the project planning and help allocate the team’s resources).\newline

The work breakdown structure should show a hierarchical decomposition of the project phases and its components. Each main phase is at a top-level and will outline the generic parts of the software development processes. The way the WBS (Work Breakdown Structure) is developed is by starting with the end objective and subdividing each main part into manageable components in terms of size, complexity and duration. Each sub objective is to follow the 80 hour rule. This means that a subtask is not to exceed 80 hours in magnitude.  The WBS for this project is shown below in figure 3.xx.

TODO sett inn Figure 3.xx Work breakdown structure\newline

\section{Project milestone plan}

TODO: put the Gantt and burndown chart here and describe the time/sprint planning.\newline

Milestones are used as tools in project management to give the team some clear and specific goals to work towards as the project timeline moves ahead. There are several milestones throughout the project, some are large milestones like; alpha and beta versions of the software. There are also milestones related to the project report, like the midterm submitting. and final delivery. Milestones can add some value to the project scheduling when used in the right manner and when setting realistic goals. Components that are important for the milestone plan are; key dates, key deadlines and external deliveries.  The team used a combined Gantt chart with milestones noted for better visualization, and to better allocate resources for meeting the milestone goals.\newline

The planned deliveries to the customer is the following dates. 
\begin{itemize}
\item 20.02.15 First prototype on paper
\item 27.02.15 Second prototype presented in proto.io[KF2]
\item 17.03.15 First working software(alpha-version)
\item 10.04.15 Second working software(beta-version) 
\item 01.05.15 Final product
\end{itemize}

\section{Quality assurance}

According to Sommerville quality assurance is “the definition of processes and standards that should lead to high-quality products and the introduction of quality processes into the manufacturing process” [AS6]. In large systems, designed to be used in a long term perspective, quality documentation is important. However, in this project a small system was developed and Sommerville notes that a more informal approach can then be applied, focusing on “establishing a quality culture” [AS6] within the development team.\newline

Therefore, this section will describe three features considered important by the group to establish such a quality culture, and thus improving the quality of the product, namely group interaction, version controlling, and interaction with the customer. Risk management and testing are also important aspects of quality assurance, and these are discussed in separate chapters(see chapter XX for risk management and chapter XX for testing). 

\subsection{Group interaction}

As was noted in section XX.XX(Pre-study - Scrum) an important part of the scrum methodology is the close collaboration between the team members. Scrum provide some events to enhance this collaboration, for instance the daily scrum meetings, and some artifacts, such as the sprint backlog. These features of scrum was used by the group and created a framework for the process development. However, scrum does not define how the group should interact, and the group interaction consisted of more than the methods provided by scrum, for instance when doing sessions of collaborative work.\newline

In order to ensure that the scrum events and the interaction external to these events would create the desired quality culture, the team discussed and agreed upon some basic rules of engagement for the project. These rules specified how the team should create a quality process in order to create a quality product, for instance by setting ground rules for communication between the group members. The tools described in section XX.XX (Tools-communication tools) were used to facilitate the implementation of the rules. In addition, meeting minutes from every meeting was made so that every group member would be aware of the status of the project even if they were not present at the meeting.\newline

The division of the group described in section XX.XX (Project management - Scrum team and roles) meant that a member of the front end part would have more detailed information about what other front end developers were doing than what individual members of the back end part were doing, and vice versa. However, when important decisions were to be made in one part of the project or important problems had to be solved, both parts would be involved in the discussion, even if the decision did not affect their part of the project directly. An example of such a decision was the choice of colours to use in the user interface.

\subsection{Git and version controlling}

The code for this project is hosted by GitHub, a tool described in section XX.XX (Tools chapter). GitHub uses the version control system git, which makes development easier by allowing multiple local branches and thereby giving users the opportunity to try out code before committing them to the master branch. The master branch is the main branch and should only include stable code. The group chose to create two repositories, one for the back end part of the project and one for the front end part since the group also was divided in this way. Doing it this way made keeping track of branches and issues on Github clearer since these often would be at a level of detail only relevant to the developers in that part of the project. 

\subsection{Customer interaction}

In section XX.XX(Pre-study-scrum) it was noted that the project was not clearly defined at the start. This meant that good customer interaction was critical, so that the group and the customer would be in agreement of what was expected of the end product. Communication with the customer was done by weekly meetings (some weeks were skipped in the later part of the project period because there were no issues to discuss), mail and by a shared Dropbox folder. In addition, the customer had access to the code on GitHub. \newline

A couple of days before the weekly meetings, the group would add a meeting agenda to the Dropbox folder. This was done to improve the structure of the meeting and to give the participants time to consider the issues on the agenda, which in turn should increase the benefits of the meeting and increase the likelihood of making decisions. Making decisions and coming to an agreement with the customer on key issues was considered important to push the project forward. After the meeting the group would add a minute of the meeting to Dropbox, so that the customer would know if all the participants had a similar understanding of the issues discussed and the decisions made. Communication by mail was mainly used to rescheduling of meetings and by the customer to give additional information to the group. 

\cleardoublepage