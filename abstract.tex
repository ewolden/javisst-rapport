\section*{\Huge Abstract}
\addcontentsline{toc}{chapter}{Abstract}
$\\[0.5cm]$

This report describes a software development project completed by seven students in the IT2901 Informatics Project II course at Norwegian University of Science and Technology (NTNU), during the spring 2015. During this project a cross-platform application that provides personalized story suggestions was developed. These recommendations are developed by use of both content-based and collaborative filtering. In addition to accurate suggestions, an additional focus point was application usability. The motivation behind this project was the assumption that personalization can lead to an increased interest in cultural heritage. The result of our work will be used to experiment with the personalization of cultural stories, to find out if this assumption holds true. The developed application was named "Vettu hva?", and should hopefully encourage users to discover and read stories that they find interesting.\newline

The first stage during development was spent refining requirements with the customer. A study of applications with a similar functionality is presented. Both of these elements were used to plan what the user interface would look like. The presented user interface is the result of several rounds of user testing and prototyping done to achieve a streamlined experience. The application and all its development stages are presented. The tools and technologies used are described with justifications for the usage of them.\newline 

The resulting application is able to recommend stories to a user in an esthetically pleasing manner. In addition to recommending stories, users can create and manage bookmarks as they see fit, and rate stories after reading them. The application achieved what it was meant to do, which was to serve as a tool for evaluating personalization techniques as a method to increase interest in cultural heritage. As the application was developed for a research trial, some of the non-functional requirements were relaxed since personalization functions were the main focus. We feel that the application meets its intended goal both functionally and esthetically.\newline

\cleardoublepage