\section*{\Huge Abstract}
\addcontentsline{toc}{chapter}{Abstract}
$\\[0.5cm]$

This report describes a software development project completed by seven students in the IT2901 Informatics Project II course at Norwegian University of Science and Technology (NTNU), during the spring 2015. During this project a cross-platform application that provides personalized story suggestions was developed. These recommendations are developed by use of both content based, and collaborative filtering. In addition to accurate suggestions, an additional focus point was application usability. The motivation behind this project was to increase the interest in cultural heritage through use of innovative technologies. The developed application was named "Vettu hva?", and should hopefully encourage users to find and read stories that they find interesting.\newline

The first stage during development was spending time refining requirements with the customer. A study of applications with a similar functionality is presented. Both of these elements were used to plan how the user interface would look like. The presented user interface is the result of several rounds of user testing and prototyping done to achieve a streamlined experience. The application and all its development stages are presented. The tools and technologies used are described in an educational way.\newline 

The resulting application is able to recommend stories to a user in an esthetically pleasing manner. In addition to recommending stories, users can create and manage bookmarks as they see fit, and rate stories after reading them. The application achieved what it was meant to do, which was to serve as a tool for testing whether this kind of personalization may lead to an increased interest in cultural heritage. The scope of the application was limited, as it was mainly designed to be used by a controlled group of users for testing purposes. We feel that the application meets this goal both functionally and esthetically.\newline

\cleardoublepage