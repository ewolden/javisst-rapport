%===================================== CHAPTER 10 Evaluation =================================

\chapter{Evaluation}

This chapter describes the final evaluation and reflection by the team for different aspects of the project. This includes positive and negative sides, what worked well and what could have been done differently.

\section{Product quality}

We are satisfied with the functional and aesthetic aspects of the application. The customer also expressed that they were happy with the final result. Regarding the requirements described in \textbf{Appendix \ref{app:requirements}}, we completed all the requirements except for the ones that were down-prioritized because of time constraints.
The application passed all the unit tests and system tests as shown in \textbf{Appendix \ref{app:unittest}} and \textbf{Appendix \ref{app:systemtest}} respectively. The system can still be further developed in the future, but the group has concluded that we are pleased with what has been accomplished on the app during these 6 months of work.

\section{Development process}

We used the agile development methodology Scrum, which has helped our work. The frequent meetings were efficient to coordinate our work and share progress, which was a necessity in a group of 7 members. Scrum is also well suited to respond to requirement changes, which was relevant in this case as the requirements for our application had to be reevaluated and changed several times over the course of the project.

\section{Project management}

Project management in this section details how we planned and followed up our project, how well we complied with the process model, and how effectively we distributed work.\newline

We could have estimated time better, as it was significantly off and some tasks proved bigger or smaller than we initially thought. It caused moments of stress when we spent too long on some tasks and didn't delegate enough time, which meant we had to down-prioritize other tasks.\newline

It was hard to use the Scrum sprints effectively. The sprints didn't always have concrete endings, and sometimes it was hard to decide what tasks to work on next. Occasionally we ended up just planning to research instead of specific tasks. We could have done more planning in between sprints and specified the tasks more clearly.\newline

The role distribution has worked well. Every person performed well in their tasks and we finished the tasks that we wanted to. By dividing up the tasks to small packages, it was possible for each person to work on a specific task at a time, while not having to worry about other factors.\newline

The requirements have been subject to many changes, and because of this, Scrum has worked quite well for us. The process has allowed us to adapt and change the application as needed to reflect the changes in the requirements.

\section{Team}

We had meetings often and kept each other up to date about status, and maintained good communication in general. We agreed about most things and we set ground rules at the beginning for how to work together. We were able to coordinate with each other so we could work independently and still have minimal problems. Everyone contributed and took initiative to get work done. Everyone was quick to point out issues and we took initiative to solve problems as fast as possible.\newline

On the other hand, we divided up the work tasks, for example into front-end and back-end, and we stuck with it that way. This is not a problem in itself, but it meant that the back-end staff did not know much about how the front-end was implemented, and front-end staff did not know much about how the back-end worked. The level of individuality has been maybe too high at times because of this.\newline

Sometimes the creativity stagnated in the group, even though it wasn't necessarily anyone's fault. We had to go back and forth a lot to come up with some solutions. Examples of this were coming up with the name of the application, and also figuring out creative ways to make the user interface user-friendly. On the positive side, the members in the group have been able to come up with good solutions when coding, that have worked well in the end. By criticizing and testing a lot, we have been able to reach good solutions for the implementation. It has been easy to bounce ideas off each other and make decisions as a team.\newline

Cooperation inside each group (front-end and back-end) has been very good. The members of front-end worked well together, and the members of the back-end also worked well together. The response from other group members when one person had a trouble has also been very good, and issues has been resolved quickly because of the tight and efficient communication.\newline

Communication between front-end and back-end staff could have been better. One group did not always know much about what the other did until very late in the project. Even though we did have meetings to share progress, when it came to the code specifics, each group was largely unaware of how the other group had implemented it. The team members could also have been better at communicating when they would not be available for meetings, or show up late. Many meetings were unnecessarily delayed because of this.\newline

In our group we had similar competencies, but different levels of experience. It strengthened our group in the way that some members has special experiences that were useful to perform certain tasks. For example, Docker and Linux experience, as well as database knowledge and the front-end framework used, which some members were able to get much experience in by working with the framework and tools.

\section{Customer interaction}

The customer meetings have been frequent (one meeting every week) and this has been useful. We had many things to discuss, and issues we had to reach an agreement on with the customer. The communication was sometimes difficult, because we did not always understand what the customer wanted, and they did not always understand our thoughts and ideas on how to solve problems.\newline

It has been useful for us to prepare for the customer meetings by writing an agenda before each meeting, and discuss in the group what we wanted to talk about with the customer. On the other hand, we could have been better at coming to an agreement inside the group about issues before each meeting. Occasionally we went to a meeting and team members gave ideas and opinions that not all the other team members understood or agreed with.\newline

We could also have been better at clearly stating our ideas about our solutions to the customer.  There has been several incidents where the customer did not understand or agree with our solutions, which cost us time as we had to reevaluate parts of the design.

\section{Limitations}

Before the start of the project, several constraints were identified in advance, and were described in \textbf{Section \ref{sec:assumptions}}. These factors as well as others limited the group in various ways.

\begin{itemize}
\item Time was the most prominent limiting factor. For 7 people with an estimated 20 hours of work each week per person, it was not possible to implement every functionality and idea that we wanted to include. This limitation was handled by using scrum to effectively organize our work, as well as making a prioritized list of requirements to make sure that the most important parts of the application would be completed first.

\item The framework limited us in some way. As discussed in \textbf{Section \ref{subsec:user_interface}} there were some of our early design choices that proved to be very difficult to implement with the chosen framework. This was handled by redesigning the interface in a way that could more easily be implemented with the framework. 

\item The experience and knowledge in the team was somewhat of a limiting factor. \textbf{Table \ref{Tab:team}} shows the team's background competencies and also shows that there was little experience with mobile application development. We also had no experience with recommender engines, which meant that in order to bypass this limitations we had to allocate some time to research the topics that we lacked knowledge in. This limitation therefore led to a further restriction to our time.

\todo{dårlig supervisor feedback som et punkt her?}
\end{itemize}

\section{Lessons learned}

Throughout the project, we discovered multiple things that we could have done differently. Here are the top three lessons we learned from this subject.

\begin{itemize}
\item We have learned that it is important to plan the work for each iteration properly. This ended up being a bit confusing when we did not always know what tasks to work on for each sprint.

\item It is important to evaluate the requirements often, and make sure that we have understood them properly. They need to be referred to often while implementing the application, and they need to be assessed and discussed during customer meetings so that they are understood the same way by the group and the customer.

\item We should have researched how to implement the personalization techniques earlier. This was started very late in the process and took up a lot of time. The reason it was delayed to begin with was because we wanted other functionality to be implemented first, but in retrospect it should have been started on earlier than we did. The necessary level of research needed for this was underestimated.
\end{itemize}

\cleardoublepage